%% $Id: format.tex,v 1.5 2013/01/30 18:20:19 evarose Exp $
%% LaTeX declarations for 
%% Some normal abbreviations.
%%
%% We use UTF-8 characters.
\usepackage[utf8]{inputenc}
%%
%% Declarations.
\usepackage{amssymb,amsmath,stmaryrd} % so we can do math
\allowdisplaybreaks             % so we can have multi-page definitions.
\usepackage{url,doi}            % so we can use proper references
%%
\usepackage{makeidx}            % definerer \index
\newcommand{\ix}[2][]{{#2\ifx&#1&{\index{#2}}\else{\index{#1#2}}\fi}}
%%
%% The kinds of theorems.
\usepackage{spamsthm} 
\spamsnewtheorem{haskell}{Haskell}
\spamsnewtheorem{discrepancy}{Discrepancy}
%%
%% Some abbreviations for the inference rules.
\DeclareRobustCommand{\lambdaJ}{\lambdaJx}
\newcommand{\lambdaJx}{\ensuremath{\lambda_{\textup{J}}}\xspace} % lambda_J
\newcommand{\BigG}{{\mathcal{G}}}                         % curly G
\newcommand{\triple}[1]{\left\langle{#1}\right\rangle}    % <...> brackets
\newcommand{\kw}[1]{{\textsf{\textbf{#1}}}}               % keyword
\newcommand{\nl}{\texttt{\char`\\n}}
%%
%% Some abbreviations.
\usepackage{xspace}
\newcommand{\etal}{\textit{etal.}\xspace}
\newcommand{\ie}{\textit{i.e.}\xspace}
\newcommand{\cf}{\textit{c.f.}\xspace}


%% Code.
\usepackage{verbatim}
\usepackage{listings}
\usepackage{xcolor}
\definecolor{gray_ulisses}{gray}{0.55}
\definecolor{castanho_ulisses}{rgb}{0.71,0.33,0.14}
\definecolor{preto_ulisses}{rgb}{0.41,0.20,0.04}
\definecolor{green_ulises}{rgb}{0.2,0.75,0}
\lstdefinelanguage{HaskellUlisses}{
  alsoletter={_},
  basicstyle=\sffamily,flexiblecolumns=true,
  sensitive=true,
  morecomment=[l][\color{gray_ulisses}\rm]{--},
  morecomment=[s][\color{gray_ulisses}\rm]{\{-}{-\}},
  morestring=[b]",
  morestring=[m]',
  stringstyle=\texttt,showstringspaces=true,
  showspaces=false,breaklines=true,showtabs=false,
  numbers=right,numberstyle=\tiny,numberblanklines=true,firstnumber=last,
  emph={[1]
    FilePath,IOError,abs,acos,acosh,all,and,any,appendFile,approxRational,asTypeOf,asin,
    asinh,atan,atan2,atanh,basicIORun,break,catch,ceiling,chr,compare,concat,concatMap,
    const,cos,cosh,curry,cycle,decodeFloat,denominator,digitToInt,div,divMod,drop,
    dropWhile,either,elem,encodeFloat,enumFrom,enumFromThen,enumFromThenTo,enumFromTo,
    error,even,exp,exponent,fail,filter,flip,floatDigits,floatRadix,floatRange,floor,
    fmap,foldl,foldl1,foldr,foldr1,fromDouble,fromEnum,fromInt,fromInteger,fromIntegral,
    fromRational,fst,gcd,getChar,getContents,getLine,head,id,inRange,index,init,intToDigit,
    interact,ioError,isAlpha,isAlphaNum,isAscii,isControl,isDenormalized,isDigit,isHexDigit,
    isIEEE,isInfinite,isLower,isNaN,isNegativeZero,isOctDigit,isPrint,isSpace,isUpper,iterate,
    last,lcm,length,lex,lexDigits,lexLitChar,lines,log,logBase,lookup,map,mapM,mapM_,max,
    maxBound,maximum,maybe,min,minBound,minimum,mod,negate,not,notElem,null,numerator,odd,
    or,ord,otherwise,pi,pred,primExitWith,print,product,properFraction,putChar,putStr,putStrLn,quot,
    quotRem,range,rangeSize,read,readDec,readFile,readFloat,readHex,readIO,readInt,readList,readLitChar,
    readLn,readOct,readParen,readSigned,reads,readsPrec,realToFrac,recip,rem,repeat,replicate,return,
    reverse,round,scaleFloat,scanl,scanl1,scanr,scanr1,seq,sequence,sequence_,show,showChar,showInt,
    showList,showLitChar,showParen,showSigned,showString,shows,showsPrec,significand,signum,sin,
    sinh,snd,span,splitAt,sqrt,subtract,succ,sum,tail,take,takeWhile,tan,tanh,threadToIOResult,toEnum,
    toInt,toInteger,toLower,toRational,toUpper,truncate,uncurry,undefined,unlines,until,unwords,unzip,
    unzip3,userError,words,writeFile,zip,zip3,zipWith,zipWith3,listArray,doParse,
    insert,union,empty,delete,member,assocs
  },
  emphstyle={[1]\color{blue}},
  emph={[2]
    Bool,Char,Double,Either,Float,IO,Integer,Int,Maybe,Ordering,Rational,Ratio,ReadS,ShowS,String,
    Word8,InPacket,Data,Map
  },
  emphstyle={[2]\color{castanho_ulisses}\textbf},
  emph={[3]
    case,class,data,deriving,do,else,if,import,in,infixl,infixr,instance,let,
    module,of,primitive,then,type,where
  },
  emphstyle={[3]\color{preto_ulisses}\textbf},
  emph={[4]
    quot,rem,div,mod,elem,notElem,seq
  },
  emphstyle={[4]\color{castanho_ulisses}\textbf},
  emph={[5]
    False,Just,Left,Nothing,Right,True,Show,Eq,Ord,Num,Monad
  },
  emphstyle={[5]\color{preto_ulisses}\textit},
  basewidth={0.5em,0.45em},
  literate={*}{{$\times$}}1 {\_}{{$\_$}}1
    {\\}{{$\lambda$}}1 {++}{{+\!+}}2 {+++}{{+\!+\!+~}}4
    {\\\\}{{\char`\\\char`\\}}1 {\\n}{{\char`\\n}}1 {\\'}{{\char`\\'}}1
    {->}{{$\to$}}2 {<-}{{$\gets$}}2 {=>}{{$\Rightarrow$}}2 {-<}{{$ $}}2
    {<=}{{$\leq$}}2 {>=}{{$\geq$}}2 {==}{{$\equiv$}}2 {/=}{{$\not\equiv$}}2
    {\ .}{{$\circ$}}2 {\ .\ }{{$\circ$}}2
    {>>}{{$\gg\,$}}2 {>>=}{{$\gg\!=\,$}}3
    {||}{{$\parallel$}}2 {&&}{{\&\!\&}}3
    {∧}{{$\land$}}1 {∨}{{$\lor$}}1 {¬}{{$\lnot$}}1 {∪}{{$\cup$}}1 {⇒}{{$\Rightarrow$}}2
}[keywords,comments,strings]

\lstnewenvironment{code}{\lstset{language=HaskellUlisses}\expandafter\let\csname lst@um_\endcsname\_}{}

\makeatletter
\newcommand{\hask}{%
  \leavevmode\hbox\bgroup\upshape
    \def\lst@boxpos{b}%
    \lsthk@PreSet\lstset{language=HaskellUlisses,columns=fullflexible}%
    \sf\let\base@font=\sf \lsthk@TextStyle
    \@ifnextchar\bgroup{\afterassignment\lst@InlineG \let\@let@token}\lstinline@}
\makeatother

%% UTF8 math.

\usepackage{amssymb,stmaryrd,latexsym}

\DeclareUnicodeCharacter{00AC}{\leftguillement} % « "<<"

\DeclareUnicodeCharacter{00AC}{\ensuremath{\neg     }} % ¬ "-,"
\DeclareUnicodeCharacter{00B0}{\ensuremath{^\circ   }} % ° "oo"
\DeclareUnicodeCharacter{00B7}{\ensuremath{\cdot    }} % · ".-"
\DeclareUnicodeCharacter{00D7}{\ensuremath{\times   }} % × "xx"
\DeclareUnicodeCharacter{00F7}{\ensuremath{\div     }} % ÷ ":-"

\DeclareUnicodeCharacter{0393}{\ensuremath{\Gamma   }} % Γ "GA"
\DeclareUnicodeCharacter{0394}{\ensuremath{\Delta   }} % Δ "DE"
\DeclareUnicodeCharacter{0398}{\ensuremath{\Theta   }} % Θ "TH"
\DeclareUnicodeCharacter{039B}{\ensuremath{\Lambda  }} % Λ "LA"
\DeclareUnicodeCharacter{039E}{\ensuremath{\Xi      }} % Ξ "XI"
\DeclareUnicodeCharacter{03A0}{\ensuremath{\Pi      }} % Π "PI"
\DeclareUnicodeCharacter{03A3}{\ensuremath{\Sigma   }} % Σ "SI"
\DeclareUnicodeCharacter{03A5}{\ensuremath{\Upsilon }} % Υ "UP"
\DeclareUnicodeCharacter{03A6}{\ensuremath{\Phi     }} % Φ "PH"
\DeclareUnicodeCharacter{03A8}{\ensuremath{\Psi     }} % Ψ "PS"
\DeclareUnicodeCharacter{03A9}{\ensuremath{\Omega   }} % Ω "OM"

\DeclareUnicodeCharacter{03B1}{\ensuremath{\alpha   }} % α "al"
\DeclareUnicodeCharacter{03B2}{\ensuremath{\beta    }} % β "be"
\DeclareUnicodeCharacter{03B3}{\ensuremath{\gamma   }} % γ "ga"
\DeclareUnicodeCharacter{03B4}{\ensuremath{\delta   }} % δ "de"
\DeclareUnicodeCharacter{03B5}{\ensuremath{\epsilon }} % ε "ep"
\DeclareUnicodeCharacter{03B6}{\ensuremath{\zeta    }} % ζ "ze"
\DeclareUnicodeCharacter{03B7}{\ensuremath{\eta     }} % η "et"
\DeclareUnicodeCharacter{03B8}{\ensuremath{\theta   }} % θ "th"
\DeclareUnicodeCharacter{03B9}{\ensuremath{\iota    }} % ι "io"
\DeclareUnicodeCharacter{03BA}{\ensuremath{\kappa   }} % κ "ka"
\DeclareUnicodeCharacter{03BB}{\ensuremath{\lambda  }} % λ "la"
\DeclareUnicodeCharacter{03BC}{\ensuremath{\mu      }} % μ "mu"
\DeclareUnicodeCharacter{03BD}{\ensuremath{\nu      }} % ν "nu"
\DeclareUnicodeCharacter{03BE}{\ensuremath{\xi      }} % ξ "xi"
\DeclareUnicodeCharacter{03C0}{\ensuremath{\pi      }} % π "pi"
\DeclareUnicodeCharacter{03C1}{\ensuremath{\rho     }} % ρ "rh"
\DeclareUnicodeCharacter{03C2}{\ensuremath{\varsigma}} % ς "fs"
\DeclareUnicodeCharacter{03C3}{\ensuremath{\sigma   }} % σ "si"
\DeclareUnicodeCharacter{03C4}{\ensuremath{\tau     }} % τ "ta"
\DeclareUnicodeCharacter{03C5}{\ensuremath{\upsilon }} % υ "up"
\DeclareUnicodeCharacter{03C6}{\ensuremath{\phi     }} % φ "ph"
\DeclareUnicodeCharacter{03C7}{\ensuremath{\chi     }} % χ "ch"
\DeclareUnicodeCharacter{03C8}{\ensuremath{\psi     }} % ψ "ps"
\DeclareUnicodeCharacter{03C9}{\ensuremath{\omega   }} % ω "om"

\DeclareUnicodeCharacter{2020}{\ensuremath{\dagger }} % † "+|"
\DeclareUnicodeCharacter{2021}{\ensuremath{\ddagger}} % ‡ "++"
\DeclareUnicodeCharacter{2026}{\ensuremath{\dots   }} % … ".."

\DeclareUnicodeCharacter{2190}{\ensuremath{\leftarrow }} % ← "<-"
\DeclareUnicodeCharacter{2191}{\ensuremath{\uparrow   }} % ↑ "u>"
\DeclareUnicodeCharacter{2192}{\ensuremath{\rightarrow}} % → "->"
\DeclareUnicodeCharacter{2193}{\ensuremath{\downarrow }} % ↓ "d>"
\DeclareUnicodeCharacter{21A6}{\ensuremath{\mapsto    }} % ↦ "|>"
\DeclareUnicodeCharacter{21AF}{\ensuremath{\lightning }} % ↯ "dd"
\DeclareUnicodeCharacter{21BA}{\ensuremath{\circlearrowleft}} % ↺ "du"

\DeclareUnicodeCharacter{21D0}{\ensuremath{\Leftarrow }} % ⇐ "<="
\DeclareUnicodeCharacter{21D1}{\ensuremath{\Uparrow   }} % ⇑ "U>"
\DeclareUnicodeCharacter{21D2}{\ensuremath{\Rightarrow}} % ⇒ "=>"
\DeclareUnicodeCharacter{21D3}{\ensuremath{\Downarrow }} % ⇓ "D>"

\DeclareUnicodeCharacter{2200}{\ensuremath{\forall    }} % ∀ "?A"
\DeclareUnicodeCharacter{2203}{\ensuremath{\exists    }} % ∃ "?E"
\DeclareUnicodeCharacter{2208}{\ensuremath{\in        }} % ∈ "in"
\DeclareUnicodeCharacter{2209}{\ensuremath{\notin     }} % ∉ "ni"
\DeclareUnicodeCharacter{2212}{\ensuremath{-          }} % −
\DeclareUnicodeCharacter{2227}{\ensuremath{\land      }} % ∧ "/\"
\DeclareUnicodeCharacter{2228}{\ensuremath{\lor       }} % ∨ "\/"
\DeclareUnicodeCharacter{2229}{\ensuremath{\cap       }} % ∩ "ca"
\DeclareUnicodeCharacter{222A}{\ensuremath{\cup       }} % ∪ "cu"
\DeclareUnicodeCharacter{2260}{\ensuremath{\neq       }} % ≠ "/="
\DeclareUnicodeCharacter{2261}{\ensuremath{\equiv     }} % ≡ "=-"
\DeclareUnicodeCharacter{2262}{\ensuremath{\not\equiv }} % ≢ "=+"
\DeclareUnicodeCharacter{2264}{\ensuremath{\leq       }} % ≤ "=<"
\DeclareUnicodeCharacter{2265}{\ensuremath{\geq       }} % ≥ ">="
\DeclareUnicodeCharacter{2270}{\ensuremath{\nleq      }} % ≰ "<+"
\DeclareUnicodeCharacter{2271}{\ensuremath{\ngeq      }} % ≱ ">+"

\DeclareUnicodeCharacter{2282}{\ensuremath{\subset    }} % ⊂ "(-"
\DeclareUnicodeCharacter{2283}{\ensuremath{\supset    }} % ⊃ ")-"
\DeclareUnicodeCharacter{2286}{\ensuremath{\subseteq  }} % ⊆ "(="
\DeclareUnicodeCharacter{2287}{\ensuremath{\supseteq  }} % ⊇ ")="
\DeclareUnicodeCharacter{2284}{\ensuremath{\nsubset    }} % ⊄ "/(-"
\DeclareUnicodeCharacter{2285}{\ensuremath{\nsupset    }} % ⊅ "/)-"
\DeclareUnicodeCharacter{2288}{\ensuremath{\nsubseteq  }} % ⊈ "/(="
\DeclareUnicodeCharacter{2289}{\ensuremath{\nsupseteq  }} % ⊉ "/)="
\DeclareUnicodeCharacter{228F}{\ensuremath{\sqsubset  }} % ⊏ "[-"
\DeclareUnicodeCharacter{2290}{\ensuremath{\sqsupset  }} % ⊐ "]-"
\DeclareUnicodeCharacter{2291}{\ensuremath{\sqsubseteq}} % ⊑ "[="
\DeclareUnicodeCharacter{2292}{\ensuremath{\sqsupseteq}} % ⊒ "]="
\DeclareUnicodeCharacter{2293}{\ensuremath{\sqcap     }} % ⊓ "[u"
\DeclareUnicodeCharacter{2294}{\ensuremath{\sqcup     }} % ⊔ "[d"

\DeclareUnicodeCharacter{22A2}{\ensuremath{\vdash     }} % ⊢ "|-"
\DeclareUnicodeCharacter{22A3}{\ensuremath{\dashv     }} % ⊣ "-|"
\DeclareUnicodeCharacter{22A4}{\ensuremath{\top       }} % ⊤ "to"
\DeclareUnicodeCharacter{22A5}{\ensuremath{\bot       }} % ⊥ "bo"
\DeclareUnicodeCharacter{22A8}{\ensuremath{\vDash     }} % ⊨ "|="
\DeclareUnicodeCharacter{22B4}{\ensuremath{\trianglelefteq}} % ⊴

\DeclareUnicodeCharacter{2308}{\ensuremath{\lceil          }} % ⌈ "[^"
\DeclareUnicodeCharacter{2309}{\ensuremath{\rceil          }} % ⌉ "]^"
\DeclareUnicodeCharacter{230A}{\ensuremath{\lfloor         }} % ⌊ "[_"
\DeclareUnicodeCharacter{230B}{\ensuremath{\rfloor         }} % ⌋ "]_"
\DeclareUnicodeCharacter{2768}{\ensuremath{\llparenthesis  }} % ❨ "(("
\DeclareUnicodeCharacter{2769}{\ensuremath{\rrparenthesis  }} % ❩ "))"
\DeclareUnicodeCharacter{27C5}{\ensuremath{\lbag           }} % ⟅ "(~"
\DeclareUnicodeCharacter{27C6}{\ensuremath{\rbag           }} % ⟆ ")~"
\DeclareUnicodeCharacter{27E6}{\ensuremath{\llbracket      }} % ⟦ "[["
\DeclareUnicodeCharacter{27E7}{\ensuremath{\rrbracket      }} % ⟧ "]]"
\DeclareUnicodeCharacter{27E8}{\ensuremath{\langle         }} % ⟨ "<."
\DeclareUnicodeCharacter{27E9}{\ensuremath{\rangle         }} % ⟩ ">."
\DeclareUnicodeCharacter{27EA}{\ensuremath{\langle\!\langle}} % ⟪ "<:"
\DeclareUnicodeCharacter{27EB}{\ensuremath{\rangle\!\rangle}} % ⟫ ">:"
\DeclareUnicodeCharacter{2983}{\ensuremath{\{\!\{}} % ⦃ "{{"
\DeclareUnicodeCharacter{2984}{\ensuremath{\}\!\}}} % ⦄ "}}"

